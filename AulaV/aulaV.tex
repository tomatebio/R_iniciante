\documentclass[handout]{beamer}
%\documentclass[handout,xcolor=pdftex,dvipsnames,table]{beamer}

%\usepackage[latin1]{inputenc}
%\documentclass[ucs]{beamer}%para sistemas com ucs
\usepackage[utf8]{inputenc}
\usepackage{verbatim}
\usepackage{tikz}   %TikZ is required for this to work.  Make sure this exists before the next line
\usepackage{pgf}

 \usepackage[scaled]{helvet}
 \renewcommand*\familydefault{\sfdefault} %% Only if the base font of the document is to be sans serif
 \usepackage[T1]{fontenc}

%\usepackage[utf8x]{inputenc}%idem
\usepackage[brazil]{babel}
\usepackage{verbatim}
\usetheme{Unicamp}%
\usecolortheme{dolphin}
%\usecolortheme{seahorse}
\usefonttheme[onlysmall]{structurebold}

\title[Linguagem R]{R para iniciantes\\ Aula V Programação}
\author {Carlos Henrique Tonhatti}
\institute[Unicamp]{Universidade Estadual de Campinas}
\date{}



 \AtBeginSection[]
 {
\begin{frame}
\frametitle{Sumário}
\tableofcontents[currentsection]
\end{frame}
}




\begin{document}
%para criar a pagina de rosto
\frame{\titlepage} %inclui a front page 

%==================================================slide
\begin{frame}
  \frametitle{Dúvidas da última aula?}
\end{frame}

% cria o sumario
\begin{frame}
 \frametitle{Sumário}
 \tableofcontents[pausesections]
  \setcounter{tocdepth}{2}% profundidade do sumario 
\end{frame}
%-------------------------------------------------------
%==================================================slide


\section{Programação em R}
\begin{frame}{Como é feita a programação em R}
  Nas últimas aulas vimos:
  \begin{itemize}[<+->]
  \item A sintaxe e a lógica das funções no R;
  \item Como organizar os dados nas classes de objetos do R;
  \item Como organizar passo-a-passo uma rotina no R (\textit{script}).
  \end{itemize}
  \pause Na aula de hoje vamos ver mais algumas técnicas que auxiliam na construção de uma rotina de trabalho no R.
\end{frame}

\subsection{Funções condicionais}
\begin{frame}{Funções condicionais}

 Em muitas tarefas desejamos que o computador execute instruções diferentes dependendo de alguma condição lógica.
  
 \begin{columns}
   \begin{column}{6cm}
     \texttt{INÍCIO\\
       \hspace{10pt}   Abrir o Forno\\
       \hspace{10pt}   SE(Forno== acesso)\\
       \hspace{10pt}\hspace{10pt}       Botar lenha\\
       \hspace{10pt}   CASO CONTRÁRIO\\
       \hspace{10pt}\hspace{10pt}       Acender fogo\\
       \hspace{10pt}   Assar pão\\
       FIM\\}
   \end{column}
   \begin{column}{6cm}
  \texttt{ \\   AbraOForno()\\
      \hspace{10pt} if(Forno==acesso)\\
     \{\\
      \hspace{10pt} ColocarLenha()\\
      \} else \{\\
     \hspace{10pt} AcenderOForno()\\
 \}\\
Assar o pão}

   \end{column}
 \end{columns}
\end{frame}

\begin{frame}{Funções condicionais --- sintaxe}

  \begin{block}{Sintaxe}
\# Executa apenas se condição== TRUE \\
if(condição)\{ \\
\hspace{10pt}  bloco de instruções\\
\}\\ \vspace{10pt}
\# Executa bloco 1 se condição== TRUE OU bloco 2 caso contrário.\\
if(condição)\{\\
\hspace{10pt}  bloco 1\\
\}else\{\\
\hspace{10pt}   bloco 2\\
\}\\    
  \end{block}
  \pause A condição pode ser qualquer regra lógica.
\end{frame}

\begin{frame}{Exemplos}
\texttt{\#par ou impar?\\
a<-9\\
if(a\%\%2==0)\{\\
\hspace{10pt}  print("É par")\\
  \}else\{\\
\hspace{10pt}   print("É impar")\\
  \}\\}
\pause
\texttt{~[1] "É impar"}
 \end{frame}
 \begin{frame}{\textit{ifelse}}
   \begin{block} {Condicional mais simples}
\# Retorna um vetor com os valores indicados\\
     ifelse(condição,valor se verdadeiro, valor se falso)\
   \end{block}
\pause
\texttt{numero<- c(0,1,2,3,4,5,6,7,8)\\
ifelse(numero\%\%2==0,"par","impar")\\
~[1] "par" ~  "impar"~ "par" ~  "impar"~ "par" ~  "impar"~ "par" \\ ~[8]~  "impar"~ "par" } 
 \end{frame}
\subsection{Funções de repetição}
\begin{frame}{Funções de repetição}
Em algumas tarefas desejamos que um bloco de instruções repetidamente até que uma condição seja satisfeita.
   \begin{columns}
     \begin{column}{6cm}
       INÍCIO\\
    Turma <- NotasAlunos\\
      CalcularMedia(Aluno1)\\
      CalcularMedia(Aluno2)\\
      CalcularMedia(Aluno3)\\
      \dots\\
      FIM\\
     \end{column}
     \begin{column}{6cm}
      INÍCIO\\
      Turma<-NotasAlunos\\
      Repita i de 1 até Numero de alunos\\
      \hspace{10pt} CalcularMedia(Alunoi)\\
       FIM     \\
    \end{column}
  \end{columns}
\end{frame}

\begin{frame}{Funções de repetição}
  \begin{block}{Funções de repetição}
    \# Para cada repetição o objeto $i$ recebe um valor da sequência. Termina no último elemento da sequência.\\
for(i in sequência)\{\\
bloco de instruções \\
\}\\ \vspace{10pt}
 \# Testa se a condição for verdadeira executa o bloco de instruções. Repete até a condição se tornar falsa.\\
 while(condição)\{\\
 bloco de instruções\\
 \}
  \end{block}
\end{frame}

\begin{frame}{Exemplo}
alunos<-matrix(1:12,ncol=3)\\
numero.alunos<-dim(alunos)[1]\\
medias<-c()\\
for(i in 1:numero.alunos)\{\\
\hspace{10pt} medias[i]<-mean(alunos[i,])\\
\}

\pause  \vspace{10pt}
Neste caso poderia usar a função \texttt{apply(alunos,1,mean)}
  
\end{frame}

\begin{frame}{Exemplo}
Saber se um número é primo

\texttt{numero<-13\\
aux<-numero-1\\
 primo<-TRUE\\
 while(primo==TRUE \& aux>1)\{\\
 \hspace{30pt}      if(numero\%\%aux==0)\{\\
  \hspace{30pt}\hspace{30pt}     primo<-FALSE\}\\
  \hspace{30pt}     aux=aux-1\\
 \}
 print(primo)
 }  
\end{frame}

\begin{frame}{Cuidado com repetições}

Função que repetem instruções \textit{``loops''} podem trazer uma série de problemas, entre eles:
\begin{itemize}[<+->]
\item ``\textit{loops} infinitos'' a repetição nunca para. A condição lógica sempre é verdadeira;
\item  ``Nunca acontece''. A condição lógica nunca é verdadeira;
\item  ``Big loops''. Quantidade de dados muito grande. Demora muito pra rodar. 
\end{itemize}
\pause
Soluções:\\
\begin{itemize}
\item Verifique atentamente as condições lógicas;
\item Tente outras funções: apply, tapply, foreach \dots \footnotemark{} \footnotetext{Assunto da próxima aula.}{}
\end{itemize}
  
\end{frame}
\subsection{Construindo funções}
\begin{frame}{Definição}

  \begin{block}{Função}
É uma sequência de instruções que fazem uma tarefa especifica  empacotada como uma unidade.
 \end{block} \pause
 Dependendo da linguagem pode receber outros  nomes: procedimento, rotina, subrotina, subprograma, método. 

\end{frame}

\begin{frame}{Classe \texttt{function} no R}
No R funções são objetos da classe \texttt{function}.\\
\texttt{> class(plot)\\
~[1] "function''}
  \pause \\ \vspace{20pt}
Podemos criar funções no R de modo parecido como criamos outros objetos.
\end{frame}

\begin{frame}{Criando uma função}
  \begin{block}{Sintaxe da função}
   nome<-function(argumentos da função)\\
\hspace{50pt}\{\\
  \hspace{50pt}   Instruções \\
  \hspace{50pt} return()
\\ \hspace{50pt}\}
    
  \end{block}
\end{frame}

\begin{frame}{Exemplo}
  \begin{columns}
    \begin{column}{6cm}
      \texttt{numero<-13\\
aux<-numero-1\\
 primo<-TRUE\\
 while(primo==TRUE \& aux>1)\{\\
 \hspace{30pt}      if(numero\%\%aux==0)\{\\
  \hspace{30pt}\hspace{30pt}     primo<-FALSE\}\\
  \hspace{30pt}     aux=aux-1\\
 \} \\
 print(primo)
 } 
    \end{column}\pause
    \begin{column}{6cm}
      \texttt{>Ehprimo<-function(numero)\{\\
aux<-numero-1\\
 primo<-TRUE\\
 while(primo==TRUE \& aux>1)\{\\
 \hspace{30pt}      if(numero\%\%aux==0)\{\\
  \hspace{30pt}\hspace{30pt}     primo<-FALSE\}\\
  \hspace{30pt}     aux=aux-1\\
 \}
 return(primo)\\
\}\\ \vspace{10pt}
> Ehprimo(13)\\
~[1] TRUE
 } 
    \end{column}
\end{columns}
 
\end{frame}
\begin{frame}{Exemplo}
\texttt{media <-function(x)\\
\hspace{30pt} \{\\
\hspace{30pt} soma=sum(x)\\
\hspace{30pt} nobs=length(x)\\
\hspace{30pt} media=soma/nobs\\
\hspace{30pt} return(media)\\
\hspace{30pt}\}}
  
\end{frame}
\begin{frame}{Uma função é um objeto}

Como qualquer objeto criado uma função é listada no espaço de trabalho.
\texttt{>ls()\\
~[1] Ehprimo media  \dots}\\ \vspace{10pt}
\pause Da classe \texttt{function}

\texttt{> class(Ehprimo)\\
~[1] "function"}
  
\end{frame}

\begin{frame}{Escopo dos objetos de uma função}

Objetos criados dentro de uma função só existem dentro da mesma função. 

\begin{columns}
  \begin{column}{6cm}
    \texttt{> media <-function(x)\\
      \hspace{30pt} \{\\
      \hspace{30pt} soma=sum(x)\\
      \hspace{30pt} nobs=length(x)\\
      \hspace{30pt} media=soma/nobs\\
      \hspace{30pt} return(media)\\
      \hspace{30pt}\} \\
      > media(c(3,4,5))\\
      ~[1] 4 \\
      > soma\\
      Erro: objeto 'soma' não encontrado} \pause
  \end{column}

  \begin{column}{6cm}
    \texttt{> mediaEsoma <-function(x)\\
      \hspace{30pt} \{\\
      \hspace{30pt} soma=sum(x)\\
      \hspace{30pt} nobs=length(x)\\
      \hspace{30pt} media=soma/nobs\\
      \hspace{30pt} return(c(media,soma))\\
      \hspace{30pt}\} \\
      > mediaEsoma(c(3,4,5))\\
      ~[1]  4 12\\
 \# Não criou o objeto ``soma'' apenas retornou o valor.}
   
  \end{column}
\end{columns}
\end{frame}
\begin{frame}{Escopo dos objetos de uma função}
Objetos criados fora de uma função estão disponíveis para uma função: 

\texttt{> soma<-sum(c(3,4,5))\\
>media2 <-function(x)\\
\hspace{30pt} \{\\
\hspace{30pt} nobs=length(x)\\
\hspace{30pt} media=soma/nobs\\
\hspace{30pt} return(media)\\
\hspace{30pt}\}\\
media2(c(3,4,5))\\
~[1] 4}
\pause 

Cuidado! Quando criar uma função sempre declare os argumentos entre (). Esperar que um objeto exista fora da função é fonte de problemas.
\end{frame}




\section{Erros de programação}
\begin{frame}{Erros no programa}
Encontrar um erro em um programa é algo bem comum. Nem sempre o  programa emite um sinal de alerta. O usuário precisa estar atento para detectar e corrigir erros.
  
\end{frame}
\subsection{Correção de erros}
\begin{frame}{Depuração do código}
 Do inglês ``\textit{Debugging}'',
 \begin{itemize}
 \item  Remover impurezas, sujidade ou imperfeições; tornar(-se)
   puro. = LIMPAR, PURIFICAR

   \item Limpar(-se) de falhar moral. = EXPURGAR, PURIFICAR

   \item  Tornar(-se) melhor em alguma coisa. = APERFEIÇOAR,
   APRIMORAR, APURAR

   \item Detectar e eliminar erros em programa de computador
 \end{itemize}
.

\textit{ in Dicionário Priberam da Língua Portuguesa [em linha], 2008-2013, http://www.priberam.pt/dlpo/depurar [consultado em 17-07-2014]}.  

\end{frame}

\begin{frame}{Princípios de depuração}
  \begin{quote}
    Beware of bugs in the above code, I have only proved it correct, not tried it.\\
    \begin{flushright}
      Donald Knuth
    \end{flushright}
  \end{quote}
  \begin{enumerate}
  \item Confirme
  \item Comece pequeno
  \item Depure de forma modular
  \end{enumerate}
  \begin{flushright}
    In: Norman Matloff. The Art of R programming,\\ No Starch press, 2011
  \end{flushright}
\end{frame}

\begin{frame}{Confirmação}
 
Confirme \textit{um-por-um} cada aspecto que você considera verdade sobre o código se \textit{é realmente} verdade.

\texttt{> a<- 10 \\
> b<-20 \\
> a<-b\\
> b<-a\\ \pause
> b \\
~[1] 20}
  \end{frame}

  \begin{frame}{Começar pequeno}
Quando iniciar o processo de depuração faça testes pequenos com casos simples. Teste cada função ou trecho do código. \\ \pause

Depois teste com conjuntos de dados maiores e vários funções ao mesmo tempo. 
   \end{frame}

   \begin{frame}{Depure de forma modular}

     \begin{itemize}[<+->]
     \item Escreva o código em forma de módulos e depois depure cada
       módulo.
     \item Faça funções que executem cada módulo, teste cada uma
       em separado.
     \item Faça um código principal curto com apenas os
       passos principais, deixe as funções que você criou em outro
       arquivo.
     \end{itemize}
\pause

\texttt{source(funcoes.r)\\
 dados<-read.table(\dots)\\
 permutacao<-teste.permutacao(dados)\\
 chi<- testdochi(dados)\\
 resultado<-list(permutacao,chi)\\
 write.table(\dots)
}
     
   \end{frame}
\subsection{Ferramentas de depuração}
\begin{frame}{Ferramentas de depuração}
  O R possui algumas ferramentas que auxiliam na correção e detecção
  de erros.
  \begin{block} {Pare se não}
    \# Para a execução se a condição não for satisfeita\\
    stopifnot(condição)
  \end{block}

  \texttt{for(i in 1:10)\{\\
      \hspace{30pt}  cat(i)\\
      \hspace{30pt} stopifnot(i>5)\\
    \}\\
    1\\
    Erro: i > 5 is not TRUE }
\end{frame}
\begin{frame}{Ferramentas de depuração}

  \begin{block}{Ferramenta\texttt{ debug}}
  \# Marca uma função para ser depurada.\\
   debug(função)\\ \vspace{10pt}
   \# Desmarca uma função da depuração\\
   undebug(função)
  \end{block}
Toda vez que se executa uma função que está marcada para ser depurada é feita a depuração linha por linha da função.
  \end{frame}

  \begin{frame}{Exemplo}
Usando a função \texttt{media} criada na aula anterior.\\
\texttt{> debug(media)\\
> media(c(1,2,3))\\
debugging in: media(c(1, 2, 3))\\
debug em \#2: \{\\
    soma = sum(x)\\
    nobs = length(x)\\
    media = soma/nobs\\
    return(media)\\
\}\\}
\end{frame}
\begin{frame}{Exemplo (cont.)}
  \begin{columns}
    \begin{column}{6cm}
      \texttt{Browse[2]> \\
        debug em \#3: soma = sum(x)\\
        Browse[2]> \\
        debug em \#4: nobs = length(x)\\
        Browse[2]> \\
        debug em \#5: media = soma/nobs\\
        Browse[2]> \\
        debug em \#6: return(media)\\
        Browse[2]> \\
        exiting from: media(c(1, 2, 3))\\
        ~[1] 2\\
        undebug(media)}
    \end{column}
    \begin{column}{6cm}
\pause
\begin{itemize}
\item Executa linha por linha da função;
\item Possibilita ver cada objeto intermediário;
\item Digite ``Q'' para sair antes do final;
\item Não esqueça de desmarcar.
\end{itemize}
   \end{column}
  \end{columns}
  \end{frame}

  \begin{frame}{Ferramentas de depuração}

    \begin{block}{Ferramenta \texttt{browser}}
    \# Executa o \texttt{debug} uma única vez a partir de uma linha específica \\
    funcao<-function(argumentos)\{\\
  \hspace{30pt}\ldots\\
  \hspace{30pt} browser()\\
  \hspace{30pt}\ldots\\
   \hspace{30pt}return( \dots)\\
\}    
    \end{block}
    
  \end{frame}
  \begin{frame}{Exemplo}
\texttt{> media <-function(x)\\
\hspace{20pt}    \{\\
\hspace{20pt}\hspace{20pt}        soma=sum(x)\\
\hspace{20pt}\hspace{20pt}        nobs=length(x)\\
\hspace{20pt}\hspace{20pt}        browser()\\
\hspace{20pt}\hspace{20pt}        media=soma/nobs\\
 \hspace{20pt}\hspace{20pt}       return(media)\\
\hspace{20pt}    \}
> media(c(1,2,3))\\
Called from: media(c(1, 2, 3))\\
Browse[1]> \\
debug em \#6: media = soma/nobs\\
Browse[2]>\\ 
debug em \#7: return(media)\\
Browse[2]> \\
~[1] 2}    
  \end{frame}


\section{Otimização de programa}

\begin{frame}{Definição}
  \begin{block}{Otimização de programa}
É um processo de modificação de um programa para que ele trabalhe de forma mais eficiente ou com menos recursos.     
  \end{block}
\pause
Um programa pode ser otimizado para que ele seja mais rápido ou capaz de operar com menos memória ou outros recursos. Aumentando a eficiência do programa.
  
\end{frame}
\begin{frame}{Avisos sobre otimização}
  \begin{quote}
    ``Premature optimization is the root of all evil.'' \\
    \begin{flushright}
      Donald Knuth
    \end{flushright}
  \end{quote}

  \begin{itemize}[<+->]
  \item Você precisa otimizar?
  \item Um código apenas 2 ou 3 vezes mais rápido pode não compensar o trabalho;
  \item Um código que seja 10 vezes mais rápido pode compensar;
  \item Antes de otimizar você precisa ter um código que funcione!
  \end{itemize}  
\end{frame}
\begin{frame}{Otimizando um código}{Evite a \texttt{repetição}}
Comandos de repetição (principalmente o \texttt{for} gastam muito tempo para serem executados. Evite-os!. Tente usar outras abordagens:
\begin{itemize}
\item Funções \texttt{apply, tapply,aggregate, etc}
\item Lembre-se que operações com vetor são feitas elemento por elemento. Não é necessário \texttt{for()} em muitas situações.
\end{itemize}
\texttt{>for(i in 1:length(numeros))\{\\
\hspace{10pt} aux[i]<-numeros[i]/2\\
\} \\ \vspace{10pt}
\# Versão mais rápida\\
aux<-numeros/2}
 
\end{frame}

\begin{frame}{Otimizando o código}{Escrevendo funções}
  \begin{itemize}[<+->]
  \item Escreva uma função que faça apenas o que você quer;
  \item Faça uma versão compilada da sua função ex: \texttt{cmpfun(função)}
   \item Escreva funções  pequenas e importantes em \texttt{C} e use elas pelo \texttt{R};
 
  
  \end{itemize}
  
\end{frame}

\begin{frame}{Otimizando o código}{Preparando os dados}
  \begin{itemize}
   \item Prefira usar matrizes ao invés de  data frame;
     \item Limpe os dados antes e então use funções mais simples ex: \texttt{sum(x)/length(x)} é mais rápido que \texttt{mean(x)} pois não tem verificação de erros.
\item Se possível, divida os dados em conjuntos menores, analise separadamente e depois junte as respostas \textit{Computação paralela}. Vários pacotes ajudam a fazer isso.
  \end{itemize}
 
\end{frame}

\begin{frame}{Medindo o tempo}

Com a função \texttt{system.time()} é possível saber quando tempo demora pra executar uma função. Use para comparar entre duas funções.\\


\texttt{>system.time(Ehprimo(4001))}\\
 \begin{tabular}{c c c}
  usuário &  sistema &decorrido \\
    0.012  &   0.000 &    0.012\\
  \end{tabular}\\ \pause
\texttt{>  system.time(Ehprimo(400001))}\\
  \begin{tabular}{c c c}
  usuário &  sistema &decorrido \\
    1.133  &   0.000 &    1.134\\
  \end{tabular}\\


  
\end{frame}


\end{document}