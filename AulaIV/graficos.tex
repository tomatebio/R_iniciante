\documentclass{beamer}
%\documentclass[handout,xcolor=pdftex,dvipsnames,table]{beamer}

%\usepackage[latin1]{inputenc}
%\documentclass[ucs]{beamer}%para sistemas com ucs
\usepackage[utf8]{inputenc}
\usepackage{verbatim}
\usepackage{tikz}   %TikZ is required for this to work.  Make sure this exists before the next line
\usepackage{pgf}

 \usepackage[scaled]{helvet}
 \renewcommand*\familydefault{\sfdefault} %% Only if the base font of the document is to be sans serif
 \usepackage[T1]{fontenc}
\usepackage{textcomp}
%\usepackage[utf8x]{inputenc}%idem
\usepackage[brazil]{babel}
\usepackage{verbatim}
\usetheme{Unicamp}%
%\usecolortheme{beaver}
\usecolortheme{dolphin}
\usefonttheme[onlysmall]{structurebold}

\title[Linguagem R]{R para iniciantes\\ Aula 5 Gráficos}
\author {Carlos Henrique Tonhatti}
\institute[Unicamp]{Universidade Estadual de Campinas}
\date{}



 \AtBeginSection[]
 {
\begin{frame}
\frametitle{Sumário}
\tableofcontents[currentsection]
\end{frame}
}

\AtBeginSubsection[]
{
   \begin{frame}
       \frametitle{Sumário}

       \tableofcontents[currentsection,currentsubsection]
       \setcounter{tocdepth}{2}
   \end{frame}
}


\begin{document}
%para criar a pagina de rosto
\frame{\titlepage} %inclui a front page 

%==================================================slide
\begin{frame}
  \frametitle{Dúvidas da última aula?}
\end{frame}

% cria o sumario
\begin{frame}
 \frametitle{Sumário}
 \tableofcontents[pausesections]
  \setcounter{tocdepth}{2}% profundidade do sumario 
\end{frame}
%-------------------------------------------------------
%==================================================slide

%criando um slide


\section{Apresentação de dados em gráficos}

\begin{frame}{Uso de gráficos}
\centering
\pgfdeclareimage[width=10cm]{hpgraphic}{hpgraphic}
\pgfuseimage{hpgraphic}  
\end{frame}

\begin{frame}{Aspectos importantes sobre gráficos}

  \begin{itemize}
  \item Deve ter algo para dizer;
  \item Deve ser legível. 
  \end{itemize}
\pause
Programas só podem ajudar na legibilidade.   
\end{frame}

\begin{frame}{Principios básicos de apresentação gráfica}
  \begin{itemize}[<+->]
  \item Ressaltar os padrões de interesse;
  \item Manter a estrutura de dados de forma que o leitor possa reconstruir os dados a partir da figura;
  \item A figura deve ter uma razão dado:tinta alta (mais dados usando a menor quantidade de tinta;
  \item A figura não deve distorcer, exagerar ou aparar os dados.
  \end{itemize}
  \vfill{} \pause 
\tiny{Ellison, A.M.(2001) \textbf{Exploratory Data Analysis and Graphic Display}. Cap.3:37--62p In:Scheiner, S. M. \& Gurevitch, J.\\\hspace{20pt} \textit{Design and Analysis of Experiments}. Oxford University Press. Oxford.}
\end{frame}

\begin{frame}{Recomendações}

  \begin{enumerate}[<+->]
  \item Não fazer gráficos tridimensionais ou coloridos a menos que seja estritamente necessário;
  \item Não colocar bordas externas nos gráficos;
  \item Não usar eixos desnecessários;
  \item Não usar linhas de grade;
  \item Não usar preenchimentos desnecessários;
  \item Não colocar título no gráfico;
  \item Usar virgulas ou  ponto nas casas decimais de acordo com o idioma;
  \item Colocar as unidades de medida na legenda dos eixos.
  \end{enumerate}
  \pause
Faça do jeito que o editor/revista pedir.
\end{frame}

\begin{frame}{Comparação}
\centering
\pgfdeclareimage[width=10cm]{estilo}{estilo}
\pgfuseimage{estilo}
  
\end{frame}
\section{Gráficos simples no R}
\subsection{Dispositivos gráficos}

\begin{frame}{Dispositivos gráficos no R}
O R possui dois tipos de dispositivos gráficos:

\begin{block}{Tela}
x11() windows() quartz()  
\end{block}
  
\begin{block}{Arquivos}
pdf() jpeg() tif() png() \ldots
\end{block}
\end{frame}

\begin{frame}{Manipulando dispositivos}

  \begin{block}{Manipulam dispositivos gráficos}
\# Retorna qual o dispositivo atual\\
dev.cur() \\ \vspace{10pt}

\# Lista os dispositivos abertos \\
dev.list() \\ \vspace{10pt}

\# Fecha o dispositivo atual\\
dev.off() \\ \vspace{10pt}   
  \end{block}
\end{frame}

\begin{frame}{Área de desenho de um dispositivo}
\centering
\pgfdeclareimage[width=10cm]{area}{area}
\pgfuseimage{area}
 \end{frame}

\subsection{Funções gráficas}
\begin{frame}{Tipos de funções gráficas}
  \begin{description}
  \item[Alto nível] desenham um novo gráfico;
  \item[Baixo nível] modificam um gráfico existente;
  \item[Interativas] adiciona ou remove informações com o mouse.
  \end{description}
\end{frame}

\begin{frame}{Funções de alto nível}
  \begin{block}{Funções gráficas de alto nível}
\# Gráficos genéricos \\ 
plot(x,y, \dots)\\ \vspace{10pt}

\# Diagrama de caixa \\
boxplot(x, \ldots)\\\vspace{10pt}

\# Histograma\\
hist(x, \ldots)\\ \vspace{10pt}

\# Gráfico de barras\\
barplot(x, \ldots)\\
\end{block}  
\end{frame}

\begin{frame}{Exemplo}
\texttt{> png("plot1.png", width=600,height=400)\\
> x<-c(1:10)\\
> y<-c(2:11)\\
> plot(x,y)\\
> dev.off()\\}
\centering

\pgfdeclareimage[width=8cm]{plot1}{plot1}
\pgfuseimage{plot1}  
\end{frame}

\begin{frame}{Detalhes função \textit{plot()}}
Argumentos da função:

\begin{description}
\item[type] altera o tipo de gráfico ( p- pontos, l - linhas, h - histograma, etc);
\item[main] título do gráfico;
\item[xlab] título do eixo x;
\item[ylab] título do eixo y;
\item[xlim] limites do eixo x;
\item[ylim] limites do eixo y;
\item[col] cores.
\end{description}
  
\end{frame}


\begin{frame}{Método de entrada usando formula}

No R é possível expressar a relação entre variáveis usando formulas.\\
\[y \sim x\] \\ 
\begin{center}
  ``A variável $y$ é dependente de $x$''
\end{center}
\texttt{plot(x,y)\\
plot(y\texttildelow x)}  
\end{frame}

\begin{frame}{Exemplos}
Vamos usar um conjunto de dados do R:\\ Medidas da qualidade do ar em Nova York, 1973.

\texttt{data(airquality)\\
names(airquality)\\
airquality\$date<-with(airquality,ISOdate(1973,Month,Day)})

\begin{table}[!htbp] \centering 
\scalebox{0.8}{
\begin{tabular}{ cccccccc} 
 & Ozone & Solar.R & Wind & Temp & Month & Day & date \\ 
1 & $41$ & $190$ & $7.400$ & $67$ & $5$ & $1$ & 1973-05-01 12:00:00 \\ 
2 & $36$ & $118$ & $8$ & $72$ & $5$ & $2$ & 1973-05-02 12:00:00 \\ 
3 & $12$ & $149$ & $12.600$ & $74$ & $5$ & $3$ & 1973-05-03 12:00:00 \\ 
4 & $18$ & $313$ & $11.500$ & $62$ & $5$ & $4$ & 1973-05-04 12:00:00 \\ 
5 & $NA$ & $NA$ & $14.300$ & $56$ & $5$ & $5$ & 1973-05-05 12:00:00 \\ 
6 & $28$ & $NA$ & $14.900$ & $66$ & $5$ & $6$ & 1973-05-06 12:00:00 \\ 
\end{tabular} }
\end{table}
\end{frame}

\begin{frame}{Exemplos}
\texttt{> plot(airquality\$date,airquality\$Ozone)\\
> plot(airquality\$Ozone $\sim$ airquality\$date)\\
> plot(Ozone$\sim$date, data=airquality)}

\centering
\pgfdeclareimage[width=8cm]{plot2}{plot2}
\pgfuseimage{plot2}
\end{frame}

\begin{frame}{Exemplos}
\texttt{> plot(Ozone$\sim$date, data=airquality, type=''l'')}
\centering
\pgfdeclareimage[width=8cm]{plot3}{plot3}
\pgfuseimage{plot3}
  
\end{frame}

\begin{frame}{Exemplos}
  \texttt{>plot(Ozone$\sim$date, data=airquality, type="h")}

  \begin{center}
    \pgfdeclareimage[width=8cm]{plot4}{plot4} \pgfuseimage{plot4}
  \end{center}

\end{frame}

\begin{frame}{Exemplos}
  \texttt{> plot(Ozone$\sim$date, data=airquality, type="h", main="Qualidade do ar",xlab="data",ylab="Ozônio",col="red")}
  \begin{center}
    \pgfdeclareimage[width=8cm]{plot5}{plot5} \pgfuseimage{plot5}
  \end{center}
\end{frame}

\begin{frame}{Funções de baixo nível}
Na posição (x, y)\\
  \begin{block}{Funções de baixo nível}

\# adiciona legenda \\
legend(x, y, legenda)\\  \vspace{10pt}

\# adiciona um eixo\\
axis(side, \ldots)\\  \vspace{10pt}

\# adiciona texto \\
text(x, y, labels, \ldots)\\  \vspace{10pt}

\# adiciona texto na margem \\
mtext(texto, side=3, line=0, \ldots)\\  \vspace{10pt}

\# adiciona pontos\\ 
points(x)\\
 \end{block}
  
\end{frame}


\begin{frame}
  \begin{block}{Funções de baixo nível}
\# adiciona linhas\\
lines(x, y)\\  \vspace{10pt}

\# adiciona segmento de reta\\
segments(x0, y0, x1, y1)\\  \vspace{10pt}

\# adiciona setas\\
arrows(x0,y0, x1, y1, angle=30, code=2)\\  \vspace{10pt}

\# adiciona linha com inclinação $b$ e intercepto $a$\\
abline(a, b)\\  \vspace{10pt}

\# adiciona linha horizontal na posição $y$\\
abline(h=y)\\  \vspace{10pt}

\# adiciona linha vertical na posição $x$\\
abline(v=x)  \vspace{10pt}
  \end{block}
\end{frame}

\begin{frame}{Exemplo}
  \begin{columns}
    \begin{column}{6cm}
\texttt{\small > plot(x,y,type="n") \\
> axis(4)\\
> text(8,8,"Comentário")\\
> mtext("texto lateral",side=4)\\
> points(1:10,rep(4,10))\\
> abline(h=7)\\
> abline(v=3)\\
> arrows(1:10,rep(4,10), \\
1:10,rep(5,10), angle=45)
}
    \end{column}

    \begin{column}{6cm}
     \pgfdeclareimage[width=5cm]{plot6}{plot6}
     \pgfuseimage{plot6}
    \end{column}
  \end{columns}
\end{frame}


\begin{frame}{Controlando detalhes}
 
A função \texttt{par} controla parâmetros gráficos. \\ \pause
Alguns argumentos:
\begin{description}
\item[cex] tamanho relativo do texto e símbolos (padrão = 1);
  \item[col] cores dos símbolos e textos (ex. col="red");
\item[pch] tipo de simbolo utilizado (ex. pch=16);
\item[new] permite que novos gráficos sejam sobrepostos (padrão FALSE)
\item[family] família da fonte (ex. ``serif'');
\item[mfrow] divide a área de plotagem  na forma de tabela (padrão c(1,1)).
\end{description}
\end{frame}

\begin{frame}{Exemplo}
 \begin{columns}
    \begin{column}{4cm}
\texttt{\small > par(mfrow=c(2,2),\\ new=FALSE,family="serif")\\ \vspace{10pt}
> plot(1:4,1:4,\\pch=19,col="green")\\ \vspace{10pt}
> plot(4:1,1:4,\\pch=22,col="red")\\ \vspace{10pt}
> plot(1:5,2:6,\\pch=25)\\ \vspace{10pt}
> plot(1:5,1:5,pch= "p"
}
    \end{column}

    \begin{column}{6cm}
     \pgfdeclareimage[width=6cm]{plot7}{plot7}
     \pgfuseimage{plot7}
    \end{column}
  \end{columns}
  
\end{frame}

\begin{frame}{Funções interativas}
Possibilitam interação com um gráfico através do mouse.

\begin{block}{Localizar um ponto}
locator(n)  
\end{block}
\texttt{> plot(Ozone$\sim$date, data=airquality)\\
> locator(n=2)\\
\$x\\
~[1] 115489104 113641840 \\ \vspace{10pt}
\$y\\
~[1] 117.0584 122.5218}
  \end{frame}

\section{Gráficos médios no R}
\subsection{Esquemas de cores}
\begin{frame}{Uso de cores}
Em alguns casos uso de cores podem ajudar a visualização dos dados \\
\texttt{> bad<-ifelse(airquality\$Ozone>=90,"orange","forestgreen")\\
> plot(Ozone$\sim$date, data=airquality,type="h",col=bad)}
\begin{center}
  \pgfdeclareimage[width=8cm]{plot8}{plot8} \pgfuseimage{plot8}
\end{center}
\end{frame}

\begin{frame}{Escolha de cores}
A escolha de quais cores usar não é algo simples. \\
Algumas ferramentas podem ajudar:
\begin{itemize}
\item \url{www.colorbrewer.org} tem esquemas de cores desenvolvidos para mapas (estão disponiveis no pacote \texttt{RColorBrewer}
\item Pacote \texttt{colorspace} tem vários esquemas de cores;
\item Pacote \texttt{dichromat} tenta mostrar o impacto dos esquemas de cores para daltônicos.
\end{itemize}
\end{frame}

\begin{frame}{Exemplo}
Pacote \texttt{RColorBrewer}\\
\centering
\pgfdeclareimage[width=6cm]{paletes7}{paletes7}
\pgfuseimage{paletes7}  
\end{frame}
\begin{frame}{Exemplo}
\texttt{> require(RColorBrewer)\\
> barplot(x,col=brewer.pal(2,"BrBG"))}

\begin{center}
  \pgfdeclareimage[width=6cm]{plot9}{plot9} 
\pgfuseimage{plot9}
\end{center}
\end{frame}
\subsection{Vários conjuntos de dados}
\begin{frame}{Dois conjuntos no mesmo gráfico}
\texttt{> x<-1:10\\
> x2<-10:1\\
> y<-1:10\\
> plot(x,y,col="red",pch=16)\\
> points(x2,y,col="blue", pch=16)\\
> legend(2,10,legend=c("azul","vermelho"), \\
pch=16,col=c("blue","red"))}
\begin{center}
  \pgfdeclareimage[width=7cm]{doisconj}{doisconj}
  \pgfuseimage{doisconj}
\end{center}
\end{frame}

\begin{frame}{Duas variáveis}
\texttt{>  par(mar=c(5,4,4,5))\\
> plot(Ozone~date, data=airquality,type="h", col="blue",ylab="Ozonio",xlab="Data")\\
> par(new=TRUE)\\
> plot(airquality\$date,airquality\$Wind,type="l", col="red",xaxt="n",yaxt="n",xlab= ``  '',ylab= ``  '' )\\
> axis(4)\\
> mtext("Vento",side=4,line=3)}\\
\end{frame}

\begin{frame}{Exemplo}
\centering
\pgfdeclareimage[width=10cm]{doiseixos}{doiseixos}
\pgfuseimage{doiseixos}
  
\end{frame}

\begin{frame}{Muitas variáveis ao mesmo tempo}
\texttt{> plot(airquality)}

\begin{center}
  \pgfdeclareimage[width=6cm]{plot10}{plot10}
 \pgfuseimage{plot10}
\end{center}
 
\end{frame}

\begin{frame}{Muitas variáveis ao mesmo tempo}
\texttt{> coplot(Ozone$\sim$Solar.R|Temp*Wind, number=c(4,4), data=airquality, pch=21)}
\begin{center}
  \pgfdeclareimage[width=6cm]{plot11}{plot11} \pgfuseimage{plot11}
\end{center}

\end{frame}
\section{Exemplos}
\begin{frame}{\textit{heatmap}}
  \texttt{> image(volcano)}
  \begin{center}
    \pgfdeclareimage[width=8cm]{plot12}{plot12}
     \pgfuseimage{plot12}
    \end{center}

\end{frame}

\begin{frame}{Pacote \texttt{Lattice}}
   \texttt{> require(lattice)\\
> wireframe(volcano)}
  \begin{center}
    \pgfdeclareimage[width=8cm]{plot13}{plot13}
     \pgfuseimage{plot13}
    \end{center}
\end{frame}

\begin{frame}{Pacote \texttt{maps}}
\texttt{> map('italy', fill = TRUE, col=brewer.pal(8,"BrBG"))}
\begin{center}
  \pgfdeclareimage[width=8cm]{plot14}{plot14} \pgfuseimage{plot14}
\end{center}
\end{frame}

\begin{frame}{Pacote \texttt{maptools} e \texttt{sp}}
\texttt{> mapa=readShapePoly("BRASIL")\\
> hidro=readShapeLines("hidrografia")\\
> plot(mapa)\\
> par(new=T)\\
> plot(hidro,col="blue")}
\begin{center}
  \pgfdeclareimage[width=6cm]{plot15}{plot15} \pgfuseimage{plot15}
\end{center}
\end{frame}



\begin{frame}{Mais exemplos}
  \begin{itemize}
  \item \url{http://rgraphgallery.blogspot.com.br/}
  \end{itemize}

  
\end{frame}

\begin{frame}{Para próxima aula}

  \begin{itemize}
  \item Ler capítulo 6 da apostila;
  \item Fazer tutorial ``Criação e edição de gráficos''
  \item Fazer exercícios ``Criação e edição de gráficos''
  \end{itemize} \pause
Mais?

\begin{itemize}
\item Leia a ajuda da função \texttt{ par};
\item Explore os sites de exemplos do slide anterior;
\item Gráficos ``avançados''
  \begin{itemize}
  \item Capítulo 7 da apostila
  \item \url{http://faculty.washington.edu/kenrice/sisg} seção 3. 
  \end{itemize}

\end{itemize}
\end{frame}
\end{document}
  
