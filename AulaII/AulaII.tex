\documentclass{beamer}
%\documentclass[handout,xcolor=pdftex,dvipsnames,table]{beamer}

%\usepackage[latin1]{inputenc}
%\documentclass[ucs]{beamer}%para sistemas com ucs
\usepackage[utf8]{inputenc}
\usepackage{verbatim}
\usepackage{tikz}   %TikZ is required for this to work.  Make sure this exists before the next line
\usepackage{pgf}

 \usepackage[scaled]{helvet}
 \renewcommand*\familydefault{\sfdefault} %% Only if the base font of the document is to be sans serif
 \usepackage[T1]{fontenc}

%\usepackage[utf8x]{inputenc}%idem
\usepackage[brazil]{babel}
\usepackage{verbatim}
\usetheme{Unicamp}%
\usecolortheme{dolphin}
%\usecolortheme{seahorse}
\usefonttheme[onlysmall]{structurebold}

\title[Linguagem R]{R para iniciantes\\ Aula II \\ Funções matemáticas}
\author {Carlos Henrique Tonhatti}
%\institute[Unicamp]{Universidade Estadual de Campinas}
\date{}



 \AtBeginSection[]
 {
\begin{frame}
\frametitle{Sumário}
\tableofcontents[currentsection]
\end{frame}
}

\AtBeginSubsection[]
{
   \begin{frame}
       \frametitle{Sumário}

       \tableofcontents[currentsection,currentsubsection]
       \setcounter{tocdepth}{2}
   \end{frame}
}


\begin{document}
%para criar a pagina de rosto
\frame{\titlepage} %inclui a front page 

%==================================================slide
\begin{frame}
  \frametitle{Dúvidas da última Aula?}
\end{frame}

% cria o sumario
\begin{frame}
 \frametitle{Sumário}
 \tableofcontents[pausesections]
  \setcounter{tocdepth}{2}% profundidade do sumario 
\end{frame}
%-------------------------------------------------------
%==================================================slide

%criando um slide


\section{Operações simples}
\begin{frame}
  \frametitle{Operadores}
    Assim como outras linguagens o R possui operadores aritméticos  comparativos, lógicos entre outros. 
\pause \\
 Para mais detalhes veja \texttt{?Syntax}
\end{frame}
\begin{frame}
  \begin{center}
  \frametitle{Operadores}
    \begin{block}{Operadores aritméticos}
      \begin{columns}
        \begin{column}{5cm}
          \begin{tabular}{c l}
            +& adição \\
          -& subtração\\
          * & multiplicação\\
           \^~ & potência \\
         \end{tabular}

        \end{column}

        \begin{column}{5cm}
          \begin{tabular}{c l}
          / & divisão\\
          \%\% &módulo\\
          \%/\% &divisão de inteiros\\
        \end{tabular}
        \end{column}
      \end{columns}
    \end{block}
\end{center}
\end{frame}

\begin{frame}
  \frametitle{Exemplos aritméticos}
\texttt{> 1+2 \\
~[1] 3\\
 > 3\^{}3\\
 ~[1] 9 \\
> 5/2 \\
~[1] 2.5 \\ \pause
> 5\%\%2 \\
~[1] 1 \\ \pause
> 5\%/\%2 \\
~[1] 2\\
} 
\end{frame}


\begin{frame}
  \frametitle{Operadores}
  \begin{block}{Operadores de comparação}
    \begin{columns}
      \begin{column}{5cm}
        \begin{tabular}{c l}
          \textless  & menor que  \\
          \textgreater  & maior que \\
          \textless = & menor ou igual\\
        \end{tabular}
      \end{column}


      \begin{column}{5cm}
        \begin{tabular}{c l}
          \textgreater= & maior ou igual\\
          == & igual\\
          != & diferente \\
         \end{tabular}
       \end{column}

    \end{columns}
  \end{block}
Comparam dois objetos e retornam verdadeiro ou falso (\texttt{TRUE, FALSE}).
\end{frame}

\begin{frame}
  \frametitle{Exemplos de comparação}
 
\texttt{> 1\textless 5  \\
~[1] TRUE \\ \pause
> 1\textgreater 5 \\
~[1] FALSE \\ \pause
> 1==5 \\
~[1] FALSE \\ \pause
> 1!=5\\
~[1] TRUE \\ \pause
> 5==5 \\
~[1]TRUE \\
}
\end{frame}

\begin{frame}
  \frametitle{Exemplos de comparação}
  Operadores de comparação podem ser usados sobre caracteres:

\texttt{> "casa"=="casa" \\
~[1] TRUE\\ \pause
}
\vspace{10pt}
 É usado a ordem alfabética para determinar quem é maior ou menor.

 \texttt{> "a"\textless"z"\\
~ [1] TRUE \\ \pause
 > "a"\textless"A" \\
~[1] TRUE \\ \pause
 > "aguia"\textless"zebrafish"\\
~[1] TRUE 
} 
\end{frame}



\begin{frame}
    \frametitle{Precedência entre operadores}
 Dentro de uma expressão os operadores de igual precedência são avaliados da esquerda para a direita exceto quando indicados por parênteses. 
 
 \begin{block}{Ordem de precedência}
    \ldots \^{} ~ \%\%{}~ \%/\%{}~  *{}~ /{}~ +{}~ - \textgreater {}~\textless{}~ \textgreater=~ \ldots 
 \end{block}
  Para mais detalhes e outros operadores veja \texttt{?Syntax}
\end{frame}


\begin{frame}
  \frametitle{Exemplo ``hipotenusa''}
      \centering
      \pgfdeclareimage[width=4cm]{triangulo}{triangulo}
      \pgfuseimage{triangulo}

      \[hipotenusa^{2} = catetoA^{2} + catetoB^{2} \]
      \[hipotenusa=\sqrt{catetoA^{2} + catetoB^{2}} \]
 \end{frame}

\begin{frame}
  \frametitle{Exemplo ``Hipotenusa''}
\texttt{> catetoA<- 4\\
> catetoB<- 3\\ \pause 
> quad.hipotenusa<- catetoA\^{}2 + catetoB\^{}2\\
> hipotenusa<- sqrt(quad.hipotenusa)\\ \vspace{10pt}\pause
OU \vspace{10pt}\\
> hipotenusa<- sqrt(catetoA\^{}2 + catetoB\^{}2)\\
> hipotenusa\\
~[1] 5}
\end{frame}
\begin{frame}
  \frametitle{Arredondamentos}
  \begin{block}{Funções de arredondamento}
\texttt{\# Arredonda o número até o número de casas decimais indicado (\textit{default} 0 casas)\\
round(x, digits=0)\\ \vspace{10pt}
\# Arredonda o número ``para  cima''.\\
ceiling(x)\\ \vspace{10pt}
\# Arrendonda o número ``para baixo''.\\
floor(x)}
  \end{block}
\end{frame}

\begin{frame}
\frametitle{Exemplos}
\texttt{> round(2.718282)\\
~[1] 3\\ \vspace{10pt}
> round(2.718282,digits=3)\\
~[1] 2.718\\ \vspace{10pt}
> ceiling(2.718282)\\
~[1] 3\\ \vspace{10pt}
> floor(2.718282)\\
~[1] 2}
  
\end{frame}



\section{Operações usando vetores}
\begin{frame}
  \frametitle{Criação de vetores}

  \begin{block}{Funções que criam vetores}
    \# Combina os argumentos em um vetor ou lista.\\
    c(\ldots) \\ \vspace{10pt} \pause
    \#  Gera uma sequência regular \\
    seq(from=1,to=1, by= \ldots)\\
    from:to \\\vspace{10pt} \pause
    \# Replica os elementos de um vetor \\
    rep(x, times)\\ 
  \end{block}
\end{frame}

\begin{frame}
  \frametitle{Exemplos}
\texttt{> Animal<- c(``Cachorro'', ``Gato'', ``Mosquito'')\\
> Peso<- c(1500,1000,1)\\ \vspace{10pt} \pause
> temperatura<- seq(from= 35,to=37,by=.5)\\
> temperatura\\
~[1] 35.0 35.5 36.0 36.5 37.0\\ \pause \vspace{10pt}
> notas<- 0:10 \\
> notas\\
 ~[1]  0  1  2  3  4  5  6  7  8  9 10\\ \vspace{10pt} \pause
}
\end{frame}

\begin{frame}
  \frametitle{Exemplos}
\texttt{> conceito<- rep(``A'',5)\\
> conceito\\
~[1] ``A'' ``A'' ``A'' ``A'' ``A'' \\ \vspace{10pt}
> conceitos<- rep(c(``A'',``B''),times=5)\\
> conceitos\\
~ [1] ``A'' ``B'' ``A'' ``B'' ``A'' ``B'' ``A'' ``B'' ``A'' ``B''\\ \vspace{10pt}
> conceitos<- rep(c(``A'', ``B''),each=5)\\
~[1] ``A'' ``A'' ``A'' ``A'' ``A'' ``B'' ``B'' ``B'' ``B'' ``B''
}
\end{frame}
 \begin{frame}
  \frametitle{Operações com um vetor }

Operações aplicadas a um vetor normalmente operam em cada do seus elementos.

\texttt{> notas\\
 ~[1]  0  1  2  3  4  5  6  7  8  9 10\\
> notas*2\\
~[1]  0  2  4  6  8 10 12 14 16 18 20 \\}
 \end{frame}



 \begin{frame}
 \frametitle{Operações entre vetores}
Operações aplicadas entre vetores operam pareando os elementos de ambos os vetores.

\texttt{> a<- c(1,2,3,4,5,6)\\
> b<- c(6,5,4,3,2,1) \\
> a+b \\
~[1] 7 7 7 7 7 7}
 \end{frame}

 \begin{frame}
   \frametitle{Operações entre vetores --- tamanhos diferentes}

   \begin{block}{Regra da ciclagem}
     Os elementos do vetor mais curto são repetidos sequencialmente
     até que a operação seja aplicada a todos os elementos do vetor
     mais longo.
   \end{block}

\texttt{> a<- c(1,2,3,4,5,6)\\
> b<- c(0,1) \\
> a*b \\
~[1] 0 2 0 4 0 6}
 \end{frame}

 \begin{frame}
   \frametitle{Operações entre vetores --- tamanhos diferentes}

Qaundo o vetor menor não for multiplo do vetor maior a operações e realizada mas aparece um aviso.\\
\texttt{> a<- c(1,2,3,4,5,6)\\
> b<- c(0,1,2,3)\\
> a*b\\
~[1]  0  2  6 12  0  6\\
Mensagens de aviso perdidas:\\
In a * b :\\
~  comprimento do objeto maior não é múltiplo do comprimento do objeto menor}
 \end{frame}

 \begin{frame}
   \frametitle{Hipotenusas}
Usando operações entre vetores é possível realizar a mesma operação para vários dados ao mesmo tempo.

\texttt{> catetosA<- c(2,3,4,7,8)\\
> catetosB<- c(2,5,3,6,5)\\
> hipotenusas<- sqrt(catetosA\^{}2 + catetosB\^{}2)\\
> hipotenusas\\
~[1] 2.828427 5.830952 5.000000 9.219544 9.433981}
 \end{frame}

 \begin{frame}
   \frametitle{Funções que operam sobre todo o vetor}

Algumas funções operam sobre todos os elementos do vetor.
\\
\texttt{> a< c(1,2,3,4,5,6,7,8,9)\\\vspace{10pt}
\# soma\\
> sum(a)\\
~[1] 45\\ \vspace{10pt}
\# média\\
> mean(a)\\
~[1] 5\\ \vspace{10pt}
\# diferença entre elementos\\ 
> diff(a)\\
~[1] 1 1 1 1 1 1 1 1\\ \vspace{10pt} \pause
}
 \end{frame}

 \begin{frame}
   \frametitle{Funções que operam sobre todo o vetor}
\texttt{\# soma cumulativa\\
> cumsum(a)\\
~[1]  1  3  6 10 15 21 28 36 45\\ \vspace{10pt} \pause
\# produto cumulativo \\
> cumprod(a) \\
~[1]      1      2      6     24    120    720   5040  40320 362880\\} \vspace{10pt} \pause

Outras funções que operam sobre todo o vetor: \\ \texttt{var,sd,min,max,range \ldots}
 \end{frame}
 

\section{Operações usando matrizes}
\begin{frame}
  \begin{center}
  \frametitle{Operadores}
    \begin{block}{Operações com matriz}
       \begin{tabular}{c l}
            + & soma elemento por elemento\\
            *& multiplicação elemento por elemento \\
          \%*\%& multiplicação entre matrizes \\
         \end{tabular}
     \end{block}
\end{center}  
\end{frame}

 \begin{frame}
 \frametitle{Funções com matriz}

 \begin{block}{Funções com matrizes}
 Sendo \textbf{A} uma matriz \\ \vspace{10pt} \pause
 \# Matriz transposta\\
 t(A)  \\ \vspace{10pt} \pause
 \# Diagonal  \\
 diag(A)  \\ \vspace{10pt} \pause
 \# Autovalores e autovetores \\
eigen(A)
    \end{block}
 \end{frame}

 \begin{frame}
   \frametitle{Exemplos}
\texttt{> A\\
\begin{tabular}{c c c c}
 &  [,1]&[,2] & [,3]\\
~[1,]&   1  &  4 &   7\\
~[2,] &  2  & 5 &   8\\
~[3,]  &  3 &    6 &   9\\
\end{tabular}\\ \vspace{10pt}
> t(A)\\
\begin{tabular}{c c c c}
   &  [,1]& [,2]& [,3]\\
~[1,] &   1 &   2 &   3\\
~[2,] &   4 &   5  &  6\\
~[3,] &   7 &   8  &  9\\
\end{tabular} \\ \vspace{10pt}
> diag(A)\\
~[1] 1 5 9\\}
\end{frame}

 \begin{frame}
  
\texttt{> eigen(A)\\
 \$values \\
~[1] 1.611684e+01 -1.116844e+00 -5.700691e-16\\ \vspace{10pt}
 \$vectors \\
 \begin{tabular}{c c c c}
&[,1]& [,2]& [,3]\\ 
~[1,] &-0.4645473 &-0.8829060 &0.4082483 \\
~[2,] & -0.5707955& -0.2395204 &-0.8164966 \\
~[3,] &-0.6770438 &0.4038651 &0.4082483\\
\end{tabular}}
 \end{frame}
\begin{frame}
  \frametitle{Dados faltantes}

No R dados faltantes são representados por \texttt{NA} (\textit{`` Not Available''}).

\texttt{> x<- c(2,4,5,NA)\\ 
    > mean(x)\\ \pause
    ~[1] NA \\ \pause
     > mean(x, na.rm=T)\\
~[1] 3.666667}
 \end{frame}

 \begin{frame}
   \frametitle{Valores infinitos}

No R valores infinitos são representados por \texttt{Inf}.

\texttt{>1/0 \\
~[1] Inf \\
>1/Inf \\
~[1] 0}
 \end{frame}

 \begin{frame}
   \frametitle{Valores indefinidos}

No R valores que não podem ser definidos são representados por \texttt{NaN} (``Not a Number'').

\texttt{> 0/0\\
       ~[1] NaN\\ 
       > sqrt(-1)\\
        ~[1] NaN}

 \end{frame}
\section{Operações usando outras classes}
\begin{frame}
  \frametitle{Operações com outras classes}
 Para aplicar operações e funções sobre objetos de outras classes (fatores, \textit{data frames}, listas, \textit{arrays}, etc) é necessário o uso de uma função  auxiliar.

\pause \vspace{10pt}
Existem várias funções e métodos para isso.
   
\end{frame}

\begin{frame}
  \frametitle{Algumas funções  auxiliares}
  \begin{block}{Funções auxiliares}
    \#   Aplica uma função sobre \textit{array} ou matriz.\\
    apply(X, MARGIN, FUN, \dots)     \\ \vspace{10pt} \pause
    \# Aplica uma função sobre uma lista ou vetor.\\
     lapply(X, FUN, \ldots) \\  \vspace{10pt} \pause
    \#  Aplica uma função sobre um vetor agrupado por fatores.\\
    tapply(X, INDEX, FUN,\dots)\\  \vspace{10pt} \pause
     X= objeto\\
 MARGIN = dimensão na qual será aplicada a função (1 para linha e 2 para coluna)\\
 INDEX= fator usado para agrupar\\
 FUN = função principal
  \end{block} 
\end{frame}

\begin{frame}
  \frametitle{Exemplos}
\texttt{> notas<-matrix(c(5,6,4,9,9,7,4,3,2),ncol=3)\\
> notas\\
\begin{tabular}{c c c c}
    & [,1]& [,2]& [,3]\\
~[1,] &   5  &  9 &   4\\
~[2,]  &  6 &   9  &  3\\
~[3,]   & 4&    7   & 2\\
\end{tabular}
\\ \pause \vspace{10pt}
> apply(X=notas, MARGIN=1, FUN=mean)\\
~[1] 6.000000 6.000000 4.333333
 } \\ \pause

No caso de médias por linha ou por coluna é mais fácil usar \texttt{rowMeans()} ou \texttt{colMeans()}.
\end{frame}

\begin{frame}
  \frametitle{Exemplos}
\texttt{> notas<-list(aluno1=c(5,6,8),aluno2=c(4,5))\\
> notas\\
\$aluno1\\
~[1] 5 6 8\\
\$aluno2\\
~[1] 4 5\\ \vspace{10pt}
> lapply(notas,mean)\\
\$aluno1\\
~[1] 6.333333\\
\$aluno2\\
~[1] 4.5\\}
\end{frame}
\begin{frame}
\frametitle{Exemplos}
\texttt{> idade<- c(25,26,55,37,21,42)\\
> sexo<- c(``F'', ``F'', ``M'', ``F'', ``M'', ``M'')\\
> tapply(X=idade, INDEX=sexo, FUN=mean)\\
\begin{tabular}{r r}
       F &       M \\
29.33333& 39.33333\\
\end{tabular}
} 
\end{frame}

\begin{frame}{Para a próxima aula}
  \begin{itemize}
  \item Ler o cap. 2 da apostila.
  \item Fazer o tutorial 2 ``Funções matemáticas'' da apostila.
  \item Fazer os exercícios ``Sequências númericas'' e ``lapply e sapply'' do Swirl.
  \end{itemize}
  
\end{frame}
\end{document}
  
